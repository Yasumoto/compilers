% $Header: /cvsroot/latex-beamer/latex-beamer/solutions/generic-talks/generic-ornate-15min-45min.en.tex,v 1.5 2007/01/28 20:48:23 tantau Exp $

\include{graphics}
\documentclass{beamer}

% This file is a solution template for:

% - Giving a talk on some subject.
% - The talk is between 15min and 45min long.
% - Style is ornate.



% Copyright 2004 by Till Tantau <tantau@users.sourceforge.net>.
%
% In principle, this file can be redistributed and/or modified under
% the terms of the GNU Public License, version 2.
%
% However, this file is supposed to be a template to be modified
% for your own needs. For this reason, if you use this file as a
% template and not specifically distribute it as part of a another
% package/program, I grant the extra permission to freely copy and
% modify this file as you see fit and even to delete this copyright
% notice. 


\mode<presentation>
{
  \usetheme{Warsaw}
  % or ...

  \setbeamercovered{transparent}
  % or whatever (possibly just delete it)
}


\usepackage[english]{babel}
% or whatever

\usepackage[latin1]{inputenc}
% or whatever

%Joe
%\usepackage{times}
%\usepackage[T1]{fontenc}
% Or whatever. Note that the encoding and the font should match. If T1
% does not look nice, try deleting the line with the fontenc.


\usepackage{listings}

\title[] % (optional, use only with long paper titles)
{Getting Classy}

\subtitle
{Classes: A Triangle Extension} % (optional)

\author{Joe Smith} % (optional, use only with lots of authors)
% - Use the \inst{?} command only if the authors have different
%   affiliation.

\institute[] % (optional, but mostly needed)
{
  Department of Computer Science\\
  Chapman University}
% - Use the \inst command only if there are several affiliations.
% - Keep it simple, no one is interested in your street address.

\date[] % (optional)
{Compiler Construction\\ \today}

%\subject{Talks}
% This is only inserted into the PDF information catalog. Can be left
% out. 



% If you have a file called "university-logo-filename.xxx", where xxx
% is a graphic format that can be processed by latex or pdflatex,
% resp., then you can add a logo as follows:

 \pgfdeclareimage[height=0.5cm]{university-logo}{chapman}
 \logo{\pgfuseimage{university-logo}}



% If you wish to uncover everything in a step-wise fashion, uncomment
% the following command: 

%\beamerdefaultoverlayspecification{<+->}


\begin{document}

%\begin{frame}
%  \titlepage
%\end{frame}

%\begin{frame}{Outline}
%  \tableofcontents
  % You might wish to add the option [pausesections]
%\end{frame}


% Since this a solution template for a generic talk, very little can
% be said about how it should be structured. However, the talk length
% of between 15min and 45min and the theme suggest that you stick to
% the following rules:  

% - Exactly two or three sections (other than the summary).
% - At *most* three subsections per section.
% - Talk about 30s to 2min per frame. So there should be between about
%   15 and 30 frames, all told.

\section{Introduction}
\begin{frame}{Intro}
	\begin{itemize}
	\item
	Extending the Triangle language with an extension of classes ... is ... fun? 
	\item
	This turned out to be relatively ridiculous, though not as brutal as I'd initially feared.
	\end{itemize}
\end{frame}

\subsection{History}
\begin{frame}{History}
	\begin{itemize}
	\item
	Smalltalk was one of the first mainstream examples of an implementation of classes of objects.
	\item
	It also introduced the term \alert{object-oriented programming}.
	\end{itemize}
\end{frame}

\begin{frame}{History}
	\begin{itemize}
	\item
	From there, the favorite programming language of Brian and I, Lisp, picked it up.
	\item
	With the advent of C++, the object-oriented programming paradigm took off.
	\end{itemize}
\end{frame}

\section{Triangle Language Changes}
\begin{frame}
	\begin{itemize}
	\item
	Two types of syntax:
	\item
	Class Type Denoter
	\item
	Method Call Command
	\end{itemize}
\end{frame}

\begin{frame}{ASTs}
\end{frame}

\begin{frame}[Class Type Denoter]{Syntax}
\alert{Class Type Denoter} \\
type IDENTIFIER ~ class\\
	DECLARATION\\
	end;
\end{frame}

\begin{frame}[Method Call Command]{Syntax}
\alert{Method Call Command} \\
IDENTIFIER-IDENTIFIER\\
\end{frame}

\begin{frame}{Example}
\alert{Class Type Denoter} \\
	type apple ~ class\\
		var color : Integer;\\
		proc setColor (setTo : Integer) ~ color := setTo;\\
		proc printColor () ~ putint(color)\\
		end;
\end{frame}

\begin{frame}[Method Call Command]{Syntax}
\alert{Method Call Command} \\
	y-setColor (7)
\end{frame}


\section{Triangle Implementation Changes}
\subsection{Scanner, Parser, and Abstract Syntax Trees}
\begin{frame}[Class Type Denoter]{Syntactic Analyzer}
\alert{Class Type Denoter} \\
case Token.CLASS:\\
    accept(Token.CLASS);\\
\\
    Declaration dAST = parseDeclaration();\\
    accept(Token.END);\\
    finish(typePos);\\
    typeAST = new ClassTypeDenoter(dAST, typePos);\\
  break;
\end{frame}
\begin{frame}[Method Call Command]{Syntactic Analyzer}
\alert{Method Call Command} \\
else if (currentToken.kind == Token.OPERATOR)\\
    acceptIt();\\
    Identifier iAST2 = parseIdentifier();\\
    accept(Token.LPAREN);\\
    ActualParameterSequence apsAST = parseActualParameterSequence();\\
    accept(Token.RPAREN);\\
    commandAST = new MethodCallCommand(iAST, iAST2, apsAST, commandPos);\\
finish(commandPos);
\end{frame}


\subsection{Contextual Analyzer}
\begin{frame}[Class Type Denoter]{Contextual Analyzer}
\alert{Class Type Denoter} \\
public Object visitClassTypeDenoter(ClassTypeDenoter ast, Object o)\\
  /*ast.D = (Declaration)*/\\
  ast.D.visit(this, null);\\
  return ast;
\end{frame}

\begin{frame}[Method Call Command]{Contextual Analyzer}
\alert{Method Call Command} \\
public Object visitMethodCallCommand(MethodCallCommand ast, Object o) \\
  //hm.. BS?\\
  Declaration binding = (Declaration) ast.I2.visit(this, null);\\
  if (binding == null)\\
    reportUndeclared(ast.I);\\
  else if (binding instanceof ProcDeclaration) \\
    ast.APS.visit(this, ((ProcDeclaration) binding).FPS);\\
   else if (binding instanceof ProcFormalParameter) \\
    ast.APS.visit(this, ((ProcFormalParameter) binding).FPS);\\
   else\\
    reporter.reportError("\"%\" is not a procedure identifier",\\
                         ast.I.spelling, ast.I.position);\\
  return null;
\end{frame}

\subsection{Run-Time Organization and Code Generation}

\begin{frame}{Code Generation}
\alert{Class Type Denoter} \\
public Object visitClassTypeDenoter(ClassTypeDenoter ast, Object o) \\
  int typeSize;\\
  if (ast.entity == null) \\
    typeSize = ((Integer) ast.D.visit(this, new Frame (0, 0))).intValue();\\
    ast.entity = new TypeRepresentation(typeSize);\\
    writeTableDetails(ast);\\
   else\\
    typeSize = ast.entity.size;\\
  return new Integer(typeSize);
\end{frame}

\begin{frame}{Code Generation}
\alert{Method Call Command} \\
public Object visitMethodCallCommand(MethodCallCommand ast, Object o) \\
  Frame frame = (Frame) o;\\
  Integer argsSize = (Integer) ast.APS.visit(this, frame);\\
  ast.I2.visit(this, new Frame(frame.level, argsSize));\\
  return null;
\end{frame}

\subsection{Interpretation}
\begin{frame}
	I looked into the Interpreter class, and got scared.
\end{frame}

\begin{frame}{Disassembler}
********** TAM Disassembler (Sun Version 2.1) **********\\
0:  PUSH        1\\
1:  JUMP        5[CB]\\
2:  LOAD  (1)   -1[LB]\\
3:  STORE (1)   0[SB]\\
4:  RETURN(0)   1\\
5:  JUMP        9[CB]\\
6:  LOAD  (1)   0[SB]\\
7:  CALL        putint  \\
8:  RETURN(0)   0\\
9:  PUSH        1\\
10:  LOADL       7\\
11:  CALL  (SB)  2[CB]\\
12:  CALL  (SB)  6[CB]\\
13:  POP   (0)   1\\
14:  HALT  
\end{frame}


\section{Conclusions}
\begin{frame}{Additions}
	\begin{itemize}
	\item
	I'd \alert{really} like to get inheritance rolling.
	\end{itemize}
\end{frame}

\begin{frame}{Thanks!}
	Thanks!
\end{frame}
\end{document}
